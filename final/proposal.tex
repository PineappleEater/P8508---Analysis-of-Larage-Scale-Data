
\documentclass[10pt]{article}
\usepackage[margin=0.65in]{geometry}
\usepackage{hyperref}
\usepackage{enumitem}
\usepackage{titlesec}
\setlist{nosep,leftmargin=1em}
\titleformat{\section}{\normalsize\bfseries}{\thesection}{0.3em}{}[\vspace{-0.5ex}]
\setlength{\parskip}{2pt}
\setlength{\parindent}{0pt}
\begin{document}
\begin{center}
\vspace{-2em}
{\large \textbf{Final Project Proposal: Medicaid Expansion and Adult Smoking}\\
\normalsize A Reproducible BRFSS Study (2003--2015)}\\[0.5em]
\small Xuange Liang (xl3493), Zexuan Yan (zy2654)
\vspace{-0.8em}
\end{center}

\section*{Article to Replicate}
\textbf{Citation}: Valvi NR, Vin-Raviv N, Akinyemiju T. (2019). The association of smoking and quit attempts with Medicaid expansion: Findings from the 2003--2015 BRFSS. \textit{Preventive Medicine Reports}, 16:100973.

\textbf{Study design}: The article examines ACA Medicaid expansion effects on adult smoking using BRFSS data (sample $n>$4 million adults) and survey-weighted \textbf{logistic regression} with repeated cross-sections to analyze (1) current smoking and (2) past-year quit attempts among smokers. \textbf{Why chosen}: public data with documented complex survey design; regression-centered methods covered in class; feasible scope with single data source; pedagogical value in handling 2011 BRFSS methodology change (cell-phone sampling + raking weights).

\section*{Data Sources \& Variables}
\textbf{Data source}: BRFSS (Behavioral Risk Factor Surveillance System) adults, 2003--2015 (CDC; $>$4 million observations, 50 states + DC). \textbf{Outcomes}: (1) Current smoking (yes/no), (2) past-year quit attempts among smokers. \textbf{Exposures}: State Medicaid expansion status by year; cessation-coverage barriers (prior authorization, copays). \textbf{Covariates}: Age, sex, race/ethnicity, education, income, employment, insurance, care access, state, year. \textbf{Survey design}: Sampling weights, strata, PSUs; Taylor-linearized variance; state clustering.

\section*{Analysis Plan}
\textbf{Replication}: (1) Download/merge BRFSS 2003--2015 data; apply sample restrictions to match original study. (2) Recode variables per article codebook; verify weighted descriptives. (3) Specify survey design with weights/strata/PSUs. (4) Estimate survey-weighted logistic regressions for both outcomes using difference-in-differences (expansion$\times$post-ACA interactions) with full covariate adjustment. (5) Include 2011+ indicator for methodology change; conduct sensitivity analyses. (6) Report ORs, AMEs, and 95\% CIs; compare with published results.

\textbf{Extensions}: (1) \textbf{Mechanism}: Triple interaction (expansion$\times$post$\times$"no barriers") to test policy pathways. (2) \textbf{Equity}: Stratify by income ($\leq$\$20k vs $>$\$20k) and age (18--49 vs 50+). (3) \textbf{Interpretability}: Report AMEs alongside ORs. (4) \textbf{Robustness}: Year/state FEs; alternative clustering; exclude low-response states.

\section*{Anticipated Challenges \& Mitigation}
\textbf{2011 methodology change}: BRFSS introduced cell-phone sampling/raking in 2011. \textit{Mitigation}: Include 2011+ indicator; separate pre/post-2011 analyses. \textbf{Policy definitions}: Expansion and barrier coding vary. \textit{Mitigation}: Follow article appendix; test alternates. \textbf{Survey design}: Complex weights critical. \textit{Mitigation}: Use R \texttt{survey}; verify effective $n$ and SEs. \textbf{Recoding diffs}: May vary from original. \textit{Mitigation}: Document; test robustness.

\section*{Timeline, Tools \& Deliverables}
\textbf{Timeline}: Wks 1--2: download/merge BRFSS data; apply sample restrictions; construct variables; weighted descriptives. Wk 3: estimate main replication models; compare with original results. Wk 4: extension analyses; finalize code/documentation; prepare slides. \textbf{Software}: R (\texttt{tidyverse}, \texttt{survey/srvyr}, \texttt{marginaleffects}) or Stata (\texttt{svyset}, \texttt{margins}). \textbf{Deliverables}: Replicated tables/figures, reproducible scripts, presentation deck.

\end{document}
