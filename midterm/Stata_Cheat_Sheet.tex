\documentclass[8pt,landscape,letterpaper]{article}
\usepackage[utf8]{inputenc}
\usepackage[english]{babel}
\usepackage{multicol}
\usepackage{calc}
\usepackage{ifthen}
\usepackage[landscape,margin=0.4cm,top=0.4cm,bottom=0.5cm]{geometry}
\usepackage{hyperref}
\usepackage{color}
\usepackage{listings}
\usepackage{fancyhdr}
\usepackage{xcolor}
\usepackage{tcolorbox}
\usepackage{fontawesome}

% Page setup with page numbers
\pagestyle{fancy}
\fancyhf{}
\renewcommand{\headrulewidth}{0pt}
\renewcommand{\footrulewidth}{0pt}
\cfoot{\fontsize{6pt}{6pt}\selectfont Page \thepage}
\rfoot{\fontsize{6pt}{6pt}\selectfont Stata Cheat Sheet}

% Colors
\definecolor{myblue}{RGB}{0,82,155}
\definecolor{mygreen}{RGB}{34,139,34}
\definecolor{myred}{RGB}{200,0,0}
\definecolor{mygray}{RGB}{150,150,150}
\definecolor{codebg}{RGB}{248,248,248}
\definecolor{codecomment}{RGB}{100,100,100}

% Stata keywords
\lstdefinelanguage{Stata}{
    morekeywords={clear,all,set,more,off,cd,capture,log,close,using,replace,
                  use,import,delimited,excel,sheet,firstrow,sasxport5,save,
                  list,describe,browse,codebook,summ,d,tab,nolabel,m,row,col,
                  gen,replace,destring,force,encode,decode,recode,max,min,
                  drop,keep,if,in,&,|,!,not,by,bysort,gsort,sort,duplicates,
                  append,merge,rename,reorder,reshape,long,wide,egen,rowtotal,
                  rowmean,rowmax,rowmin,mean,count,substr,strlen,strpos,string,
                  strupper,strlower,strproper,split,subinstr,concat,punct,
                  reg,regress,logit,logistic,svy,svyset,predict,margins,test,
                  xtset,xtreg,fe,pweight,vce,robust,r,cluster,i.,c.,##,#,
                  label,define,values,count,_N,_merge,display},
    morekeywords=[2]{gen,replace,encode,decode,merge,reg,logit,logistic,summ,
                     list,tab,describe,drop,keep,append,reshape,xtset,xtreg,
                     predict,margins},
    sensitive=false,
    morecomment=[l]{//},
    morecomment=[l]{*},
    morestring=[b]"
}

% Stata code style - improved highlighting
\lstdefinestyle{statastyle}{
    language=Stata,
    basicstyle=\ttfamily\tiny\color{black},
    backgroundcolor=\color{codebg},
    keywordstyle=[1]\color{myblue}\bfseries,
    keywordstyle=[2]\color{myblue}\bfseries,
    commentstyle=\color{codecomment}\itshape,
    stringstyle=\color{myred},
    numbers=none,
    breaklines=true,
    frame=lines,
    framerule=0.5pt,
    rulecolor=\color{mygray},
    xleftmargin=2pt,
    xrightmargin=2pt,
    aboveskip=2pt,
    belowskip=1pt,
    showstringspaces=false,
    tabsize=2,
    columns=fixed,
    keepspaces=true
}

\lstset{style=statastyle}

% Section formatting - more compact
\makeatletter
\renewcommand{\section}{\@startsection{section}{1}{0mm}%
                                {-0.6ex plus -.3ex minus -.1ex}%
                                {0.2ex plus .1ex}%
                                {\normalfont\fontsize{9pt}{9pt}\selectfont\bfseries\color{myblue}}}
\renewcommand{\subsection}{\@startsection{subsection}{2}{0mm}%
                                {-0.4ex plus -.2ex minus -.05ex}%
                                {0.1ex plus .05ex}%
                                {\normalfont\fontsize{7.5pt}{7.5pt}\selectfont\bfseries\color{myblue}}}
\makeatother

% Don't print section numbers
\setcounter{secnumdepth}{0}

\setlength{\parindent}{0pt}
\setlength{\parskip}{0pt}

% Reduce list spacing
\usepackage{enumitem}
\setlist{nosep,topsep=0pt,partopsep=0pt,itemsep=1pt,parsep=0pt}
\setlength{\leftmargin}{12pt}

% Custom boxes - subtle style
\newtcolorbox{warningbox}{
    colback=white,
    colframe=white,
    colbacktitle=white,
    coltitle=black,
    fonttitle=\bfseries\fontsize{6.5pt}{6.5pt}\selectfont,
    title=\faExclamationTriangle\ Warning,
    left=4pt,right=2pt,top=1pt,bottom=1pt,
    boxrule=0pt,
    leftrule=1.5pt,
    colframe=mygray,
    boxsep=2pt
}

\newtcolorbox{tipbox}{
    colback=white,
    colframe=white,
    colbacktitle=white,
    coltitle=black,
    fonttitle=\bfseries\fontsize{6.5pt}{6.5pt}\selectfont,
    title=\faLightbulbO\ Tip,
    left=4pt,right=2pt,top=1pt,bottom=1pt,
    boxrule=0pt,
    leftrule=1.5pt,
    colframe=mygray,
    boxsep=2pt
}

\begin{document}

\raggedright
\fontsize{7pt}{8pt}\selectfont
\setlength{\columnsep}{0.3cm}
\begin{multicols}{3}

% Title
\begin{center}
     \fontsize{11pt}{11pt}\selectfont\textbf{Stata Cheat Sheet for Midterm} \\
     \fontsize{6pt}{6pt}\selectfont Large Scale Data Analysis - Part 2
\end{center}\vspace{-0.3cm}

% -----------------------------------------------------------------------
\section{1. Basic Setup \& Data Import}

\subsection{Standard Setup}
\begin{lstlisting}
// Clear memory and set options
clear all
set more off

// Set working directory
cd "path/to/folder"

// Start logging
capture log close
log using "analysis.log", replace
\end{lstlisting}

\subsection{Import Data}
\begin{lstlisting}
// Import CSV
import delimited "file.csv", clear

// Import Excel
import excel "file.xlsx", ///
    sheet("Sheet1") firstrow clear

// Import SAS XPT
import sasxport5 "file.xpt", clear

// Load Stata file
use "file.dta", clear

// Save data
save "filename.dta", replace
\end{lstlisting}

\begin{warningbox}
Always use \texttt{clear} or \texttt{, clear} option when loading new data to avoid "no; data in memory would be lost" error.
\end{warningbox}

% -----------------------------------------------------------------------
\section{2. Data Exploration}

\subsection{Basic Viewing}
\begin{lstlisting}
// View first 10 rows
list var1 var2 var3 in 1/10

// View specific observations
list if condition

// Browse data (opens viewer)
browse

// Describe variables
describe
describe var1 var2

// View variable type
codebook varname
\end{lstlisting}

\subsection{Summary Statistics}
\begin{lstlisting}
// Basic summary
summ varname

// Detailed summary (with percentiles)
summ varname, d
// Shows: p1, p5, p10, p25, p50, p75,
//        p90, p95, p99

// Multiple variables
summ var1 var2 var3, d
\end{lstlisting}

\subsection{Frequency Tables}
\begin{lstlisting}
// Frequency table (include missing)
tab varname, m

// Show numeric codes (not labels)
tab varname, nolabel

// Two-way table with row %
tab var1 var2, row m

// Two-way table with column %
tab var1 var2, col m

// Both row and column %
tab var1 var2, row col
\end{lstlisting}

\subsection{Check Data Size}
\begin{lstlisting}
// Number of observations
display _N

// Number of variables
describe, short
\end{lstlisting}

% -----------------------------------------------------------------------
\section{3. Variable Creation}

\subsection{Generate New Variable}
\begin{lstlisting}
// Create empty variable
gen newvar = .

// Create with value
gen age_squared = age^2

// Create constant
gen constant = 1
\end{lstlisting}

\subsection{Replace Values}
\begin{lstlisting}
// Replace all values
replace varname = new_value

// Conditional replace
replace var = value if condition
\end{lstlisting}

\begin{warningbox}
\textbf{CRITICAL:} Always protect missing values!\\
Use: \texttt{if var != .} in conditions\\
Missing (.) is treated as infinity in Stata.
\end{warningbox}

\subsection{Binary Variables (0/1 Flags)}
\begin{lstlisting}
// Method 1: Standard approach
gen flag = 1 if condition
replace flag = 0 if !condition

// Method 2: One-liner
gen flag = (condition) if var != .

// Example: Age >= 18
gen adult = 1 if age >= 18 & age != .
replace adult = 0 if age < 18

// Example: High cost (>$50,000)
gen expensive = 1 if cost > 50000 ///
    & cost != .
replace expensive = 0 if cost <= 50000

// VERIFY binary variable
summ flag
// mean should be 0-1, min=0, max=1
\end{lstlisting}

\subsection{Categorical Variables}
\begin{lstlisting}
// Method 1: Manual creation
gen category = .
replace category = 1 if condition1
replace category = 2 if condition2
replace category = 3 if condition3

// Add value labels
label define cat_lbl 1 "Low" ///
    2 "Medium" 3 "High"
label values category cat_lbl

// Method 2: recode
recode age (0/29=1) (30/49=2) ///
    (50/max=3), gen(age_group)

// Method 3: encode string variable
encode string_var, gen(numeric_var)
\end{lstlisting}

\subsection{Top-coding (Capping Outliers)}
\begin{lstlisting}
// Create clean version
gen cost_clean = cost

// Top-code at $1,000,000
replace cost_clean = 1000000 ///
    if cost > 1000000 & cost != .

// Verify
summ cost, d
summ cost_clean, d
// Check: max of clean version = cap
\end{lstlisting}

\subsection{Logarithmic Transformation}
\begin{lstlisting}
// Handle right-skewed data
// Step 1: Add small value to avoid log(0)
gen cost_plus1 = cost + 1

// Step 2: Take log
gen log_cost = log(cost_plus1)

// Alternative: only for positive values
gen log_cost = log(cost) if cost > 0
\end{lstlisting}

\subsection{EGEN - Extended Generation}
\begin{lstlisting}
// Row operations
egen total = rowtotal(var1 var2 var3)
egen mean = rowmean(var1 var2 var3)
egen max = rowmax(var1 var2 var3)
egen min = rowmin(var1 var2 var3)

// Grouped statistics
egen mean_by_group = mean(var), ///
    by(group_var)

// Example: County-level averages
egen avg_income_county = mean(income), ///
    by(county_name)

// Count non-missing
egen count_nonmiss = count(var), ///
    by(group)

// String concatenation
egen fullname = concat(first last), ///
    punct(" ")
\end{lstlisting}

% -----------------------------------------------------------------------
\section{4. Missing Values}

\subsection{Identify Missing Values}
\begin{lstlisting}
// Count missing
count if varname == .

// Summary shows N
summ varname
// Total N vs variable N

// Show missing in table
tab varname, m
\end{lstlisting}

\subsection{Recode Missing Values}
\begin{lstlisting}
// Set specific values to missing
replace var = . if var == 99
replace var = . if var == 999
replace var = . if var < 0

// Example from BRFSS
replace height = . if height == 7777
replace height = . if height == 9999
\end{lstlisting}

\begin{warningbox}
\textbf{Common Mistake:}\\
\texttt{replace var = 100 if var > 100}\\
This will set missing to 100!\\
\textbf{Correct:}\\
\texttt{replace var = 100 if var > 100 \& var != .}
\end{warningbox}

\subsection{Create Missing Indicator}
\begin{lstlisting}
// Flag for missing
gen miss_flag = (varname == .)

// Or explicitly
gen miss_flag = 0
replace miss_flag = 1 if varname == .
\end{lstlisting}

% -----------------------------------------------------------------------
\section{5. String Variables}

\subsection{String to Numeric}
\begin{lstlisting}
// Basic conversion (force ignores errors)
destring string_var, gen(num_var) force

// Check what couldn't be converted
tab string_var if num_var == .

// Example: Handle "120 +"
destring lengthofstay, gen(los_num) force
replace los_num = 120 ///
    if lengthofstay == "120 +"
\end{lstlisting}

\subsection{Numeric to String}
\begin{lstlisting}
// Convert number to string
gen str_var = string(numeric_var)

// With formatting
gen str_var = string(num_var, "%9.2f")
\end{lstlisting}

\subsection{String to Categorical (encode)}
\begin{lstlisting}
// Create numeric with labels
encode string_var, gen(categorical_var)

// Example: Admission type
encode typeofadmission, ///
    gen(admission_type_num)

// Verify
tab typeofadmission admission_type_num
\end{lstlisting}

\subsection{String Manipulation}
\begin{lstlisting}
// Case conversion
gen upper = strupper(string_var)
gen lower = strlower(string_var)
gen proper = strproper(string_var)

// Extract substring (pos starts at 1!)
gen first5 = substr(string_var, 1, 5)
gen char2to4 = substr(string_var, 2, 3)

// Find substring position
gen pos = strpos(string_var, "keyword")
// Returns 0 if not found

// String length
gen length = strlen(string_var)

// Replace text
gen new = subinstr(string_var, ///
    "old", "new", .)
// Last argument: . = replace all

// Split string
split string_var, gen(part) parse("_")
// Creates: part1, part2, part3, ...
\end{lstlisting}

\begin{warningbox}
\textbf{substr() position starts at 1!}\\
\texttt{substr(str, 1, 2)} = first 2 chars\\
\texttt{substr(str, 2, 1)} = 2nd char only
\end{warningbox}

% -----------------------------------------------------------------------
\section{6. Data Merging}

\subsection{Merge Types}
\begin{lstlisting}
// 1:1 - Both datasets: 1 row per ID
merge 1:1 id_var using "file.dta"

// 1:M - Master: 1 row/ID, Using: many rows/ID
merge 1:m id_var using "file.dta"

// M:1 - Master: many rows/ID, Using: 1 row/ID
merge m:1 id_var using "file.dta"

// M:M - Both: many rows/ID (rare, avoid)
// Use joinby instead if needed
\end{lstlisting}

\subsection{M:1 Merge Example (Most Common)}
\begin{lstlisting}
// Merge county data to individual records
// Main: Individual hospitalizations
//       (many records per county)
// Using: County characteristics
//        (one record per county)

use "hospital_data.dta", clear

// Rename if needed to match
rename hospitalcounty County_Name

// Perform M:1 merge
merge m:1 County_Name ///
    using "county_data.dta"

// CHECK merge results
tab _merge
/*
_merge values:
  1 = master only (hospital records
      with no county match)
  2 = using only (counties with no
      hospital records)
  3 = matched successfully
*/

// Keep what you want
keep if _merge == 1 | _merge == 3
// Keeps all hospital records

// Clean up
drop _merge
\end{lstlisting}

\subsection{Merge Workflow}
\begin{lstlisting}
// Step 1: Check merge variable exists
describe merge_var

// Step 2: Check if unique (if "1" side)
duplicates report merge_var
// Should show 0 duplicates

// Step 3: Ensure variable names match
// If not, rename in one dataset
rename old_name new_name

// Step 4: Check variable types match
describe merge_var
// Both should be numeric or string

// Step 5: Perform merge
merge type merge_var using "file.dta"

// Step 6: Always check _merge!
tab _merge

// Step 7: Keep desired records
keep if _merge == 1 | _merge == 3

// Step 8: Drop _merge
drop _merge
\end{lstlisting}

\begin{tipbox}
\textbf{Quick merge type decision:}\\
Ask: "How many rows per ID?"\\
Master(M) : Using(1) $\rightarrow$ M:1\\
Master(1) : Using(M) $\rightarrow$ 1:M\\
Master(1) : Using(1) $\rightarrow$ 1:1
\end{tipbox}

\subsection{Append (Stack Datasets)}
\begin{lstlisting}
// Combine datasets with same structure
use "data2020.dta", clear
append using "data2021.dta"
append using "data2022.dta"

// All rows are kept, stacked vertically
\end{lstlisting}

% -----------------------------------------------------------------------
\section{7. Data Reshaping}

\subsection{Wide to Long}
\begin{lstlisting}
// Wide format:
// id  bp1  bp2  bp3  hr1  hr2  hr3
// 1   120  118  115  72   70   68

reshape long bp hr, i(id) j(round)

// Long format:
// id  round  bp   hr
// 1   1      120  72
// 1   2      118  70
// 1   3      115  68
\end{lstlisting}

\subsection{Long to Wide}
\begin{lstlisting}
// Long format:
// id  round  bp   hr
// 1   1      120  72
// 1   2      118  70

reshape wide bp hr, i(id) j(round)

// Wide format:
// id  bp1  bp2  hr1  hr2
// 1   120  118  72   70
\end{lstlisting}

\begin{warningbox}
\textbf{reshape} permanently changes data.\\
Always \texttt{save} before reshaping!
\end{warningbox}

% -----------------------------------------------------------------------
\section{8. Regression Analysis}

\subsection{Linear Regression}
\begin{lstlisting}
// Simple regression
reg outcome predictor

// Multiple regression
reg y x1 x2 x3

// With robust standard errors
reg y x1 x2, robust
// or
reg y x1 x2, r
\end{lstlisting}

\subsection{Categorical Variables in Regression}
\begin{lstlisting}
// Use i. prefix for categorical
reg outcome continuous_var i.category

// Example
reg totalcosts lengthofstay ///
    i.agegroup i.admission_type

// Stata automatically:
// - Creates dummy variables
// - Omits first category (reference)
// - Shows each category coefficient
\end{lstlisting}

\subsection{Continuous Variables}
\begin{lstlisting}
// Default: continuous
reg y x1 x2

// Explicit: c. prefix (optional)
reg y c.x1 c.x2
\end{lstlisting}

\subsection{Interaction Terms}
\begin{lstlisting}
// Categorical x Categorical
reg y i.var1##i.var2
// ## includes main effects + interaction

// Categorical x Continuous
// IMPORTANT: Use c. for continuous!
reg y i.category##c.continuous

// Example: Does income effect vary by sex?
reg health i.sex##c.income age

// Only interaction (no main effects)
reg y i.var1#i.var2
\end{lstlisting}

\begin{warningbox}
\textbf{Interaction with continuous:}\\
MUST use \texttt{c.} prefix!\\
Wrong: \texttt{i.sex\#\#age}\\
Right: \texttt{i.sex\#\#c.age}
\end{warningbox}

\subsection{Logistic Regression}
\begin{lstlisting}
// Binary outcome (0/1)
logistic binary_y x1 x2 i.category
// Reports Odds Ratios (OR)

// Alternative: logit (reports log-odds)
logit binary_y x1 x2 i.category

// Example
logistic expensive_stay ///
    County_Income lengthofstay ///
    i.agegroup i.ED_flag
\end{lstlisting}

\subsection{Linear vs Logistic Interpretation}
\begin{lstlisting}
// LINEAR regression on binary outcome
reg expensive_stay County_Income, r
// Coefficient: Percentage point difference
// Example: coef = 0.04
// Interpretation: "County income increase
// of $1000 associated with 4 percentage
// point increase in probability of
// expensive stay (e.g., 10% to 14%)"

// LOGISTIC regression
logistic expensive_stay County_Income
// Coefficient: Odds Ratio
// Example: OR = 1.02
// Interpretation: "County income increase
// of $1000 associated with 2% increase
// in the odds of expensive stay"
\end{lstlisting}

\subsection{Survey Weights}
\begin{lstlisting}
// Set survey design
svyset [pweight = weight_var]

// Weighted regression
svy: reg y x1 x2 i.category

// Weighted logistic
svy: logistic binary_y x1 x2

// Why use weights?
// - Make results representative
// - Account for survey design
// - Adjust for non-response
\end{lstlisting}

\subsection{Post-Regression Commands}
\begin{lstlisting}
// Test joint significance
reg y x1 x2 x3
test x1 x2
// Tests: x1 = x2 = 0

// Predicted values
predict yhat

// Residuals
predict resid, residuals

// Margins (adjusted predictions)
reg y x1 i.group
margins group
// Shows predicted y for each group
\end{lstlisting}

\subsection{Individual Fixed Effects}
\begin{lstlisting}
// For panel/longitudinal data
// Controls for all time-invariant
// individual characteristics

// Set panel structure
xtset person_id time_var

// Fixed effects regression
xtreg y x1 x2, fe

// Why use FE?
// - Within-person analysis
// - Control for unmeasured confounders
// - Stronger causal inference
\end{lstlisting}

% -----------------------------------------------------------------------
\section{9. Verification \& Validation}

\subsection{Verify Binary Variables}
\begin{lstlisting}
// Create binary flag
gen flag = 1 if cost > 50000 & cost != .
replace flag = 0 if cost <= 50000

// CHECK 1: Summary statistics
summ flag
// mean: 0-1, min: 0, max: 1, N correct?

// CHECK 2: Cross-tabulation
tab flag, m
// Should show: 0, 1, and . only

// CHECK 3: Verify cutoff
summ cost if flag == 1
// min should be > 50000
summ cost if flag == 0
// max should be <= 50000
\end{lstlisting}

\subsection{Verify Categorical Variables}
\begin{lstlisting}
// After encoding or recoding
tab old_var new_var
// Check mapping is correct

// With percentages
tab old_var new_var, row col
\end{lstlisting}

\subsection{Verify Continuous Variables}
\begin{lstlisting}
// After transformation
summ original_var, d
summ clean_var, d
// Compare: mean, min, max, N

// Grouped summary
bysort group: summ var
// or
summ var if group == 1
summ var if group == 0
\end{lstlisting}

\subsection{Check for Missing}
\begin{lstlisting}
// Count missing
count if var == .

// Identify observations with missing
list id var if var == .

// Missing by group
tab group, m
bysort group: count if var == .
\end{lstlisting}

\subsection{Verify Merge Success}
\begin{lstlisting}
// After merge
tab _merge

// List unmatched from master
list id if _merge == 1

// List unmatched from using
list id if _merge == 2

// Check merged variable
summ merged_var if _merge == 3
// Should have valid values
\end{lstlisting}

% -----------------------------------------------------------------------
\section{10. Common Workflows}

\subsection{Clean Outcome Variable}
\begin{lstlisting}
// Step 1: Explore
codebook outcome
summ outcome, d
tab outcome, m

// Step 2: Identify issues
// - Missing values?
// - Outliers?
// - Correct range?

// Step 3: Create clean version
gen outcome_clean = outcome

// Step 4: Handle missing
replace outcome_clean = . if outcome == 99
replace outcome_clean = . if outcome < 0

// Step 5: Handle outliers (top-code)
replace outcome_clean = 1000000 ///
    if outcome > 1000000 & outcome != .

// Step 6: Verify
summ outcome_clean, d
tab outcome outcome_clean, m
\end{lstlisting}

\subsection{Prepare Covariates}
\begin{lstlisting}
// Continuous variable
// - Check range
summ age, d
// - Handle missing
replace age = . if age == 99
// - Create squared term if needed
gen age_squared = age^2

// Categorical variable (numeric)
// - Check values
tab category, m
// - Create labeled version
label define cat_lbl 1 "A" 2 "B" 3 "C"
label values category cat_lbl

// Categorical variable (string)
// - Encode to numeric
encode string_var, gen(category_num)
// - Verify
tab string_var category_num

// Binary flag
// - Create 0/1
gen flag = 1 if condition & var != .
replace flag = 0 if !condition
// - Verify
summ flag
tab flag, m
\end{lstlisting}

\subsection{Complete Analysis Example}
\begin{lstlisting}
// Research Q: County income effect on
// hospitalization costs?

// Step 1: Load and check data
use "hospital_data.dta", clear
describe
summ totalcosts, d

// Step 2: Clean outcome
gen cost_clean = totalcosts
replace cost_clean = 1000000 ///
    if totalcosts > 1000000 & totalcosts != .

// Step 3: Clean covariates
encode agegroup, gen(age_num)
gen ED_flag = (ed_indicator == "Y") ///
    if ed_indicator != ""

// Step 4: Merge county data
rename county County_Name
merge m:1 County_Name ///
    using "county_income.dta"
keep if _merge == 1 | _merge == 3
drop _merge

// Step 5: Check merged data
summ County_Income, d
// Report min and max

// Step 6: Run regression
reg cost_clean County_Income ///
    lengthofstay i.age_num i.ED_flag, r

// Step 7: Interpret
// "Controlling for length of stay, age,
// and ED status, each $1000 increase in
// county income is associated with
// $XX increase in hospital costs.
// This is statistically significant
// (p<0.05)."
\end{lstlisting}

% -----------------------------------------------------------------------
\section{11. Important Reminders}

\subsection{Critical Points}

\begin{warningbox}
\textbf{Top 5 Common Mistakes:}
\begin{enumerate}
\item \textbf{Missing values:} Always use \texttt{\& var != .}
\item \textbf{substr():} Position starts at 1, not 0
\item \textbf{Interaction:} Use \texttt{c.} for continuous
\item \textbf{\_merge:} Always \texttt{tab \_merge} after merge
\item \textbf{Binary range:} Check \texttt{summ} shows 0-1
\end{enumerate}
\end{warningbox}

\subsection{Statistical Significance vs Practical}
\begin{lstlisting}
// Example: p=0.007, diff=3 minutes
// Statistical: YES (p<0.05)
// Practical: MAYBE
// - 3 min might be too small
// - But >10% relative difference
// - Large sample = "overpowered"
//   (can detect tiny differences)

// Always discuss BOTH in interpretation
\end{lstlisting}

\subsection{Regression Interpretation}
\begin{lstlisting}
// Linear regression coefficient:
// "Each 1-unit increase in X is
// associated with beta-unit change in Y,
// controlling for other variables."

// Logistic regression OR:
// "Each 1-unit increase in X is
// associated with ORx change in the
// odds of Y, controlling for others."

// Binary outcome + linear regression:
// "Each 1-unit increase in X is
// associated with beta percentage point
// change in probability of Y."
\end{lstlisting}

\subsection{When to Use What}

\textbf{Data Types:}
\begin{itemize}
\item Cross-sectional survey: Prevalence, associations
\item Longitudinal survey: Within-person changes, causality
\item Claims data: Utilization, costs, readmission
\item EHR data: Clinical details, single system
\end{itemize}

\textbf{Merge Types:}
\begin{itemize}
\item M:1: Individual records + area-level data
\item 1:M: Visits + medications per visit
\item 1:1: Same IDs in both datasets
\end{itemize}

\textbf{Regression Types:}
\begin{itemize}
\item Linear: Continuous outcome
\item Linear: Binary outcome (percentage points)
\item Logistic: Binary outcome (odds ratios)
\item Fixed effects: Panel data, within-person
\end{itemize}

\subsection{Cheat Sheet Usage Tips}
\begin{tipbox}
\textbf{During the exam:}
\begin{enumerate}
\item Start with exploration (summ, tab, describe)
\item Create clean versions of variables
\item Verify each step before moving on
\item Check merge with \texttt{tab \_merge}
\item Verify binary variables with \texttt{summ}
\item Write complete interpretations
\end{enumerate}
\end{tipbox}

\subsection{Quick Reference}

\fontsize{6pt}{6.5pt}\selectfont
\begin{tabular}{@{}ll@{}}
\hline\textbf{Task} & \textbf{Command} \\
\hline
Load data & use "file.dta", clear \\
Import CSV & import delimited "file.csv" \\
Summary stats & summ var, d \\
Frequency & tab var, m \\
Create var & gen newvar = expr \\
Binary flag & gen flag = (condition) \\
Top-code & replace var = cap if var > cap \& var != . \\
Encode string & encode str, gen(num) \\
String to num & destring str, gen(num) force \\
M:1 merge & merge m:1 id using "file.dta" \\
Check merge & tab \_merge \\
Linear reg & reg y x1 x2, r \\
With category & reg y x1 i.cat \\
Interaction & reg y i.cat1\#\#c.cont \\
Logistic & logistic binary\_y x1 x2 \\
Verify binary & summ flag (should be 0-1) \\
Cross-check & tab oldvar newvar \\
\hline
\end{tabular}
\normalsize

\vspace{0.1cm}
{\fontsize{5pt}{6pt}\selectfont \textbf{Stata Cheat Sheet v1.0 | October 2025}}

\end{multicols}
\end{document}
